\documentclass{article}
\usepackage{amsmath}
\usepackage[italian]{babel}
\usepackage[letterpaper,top=1.5cm,bottom=2cm,left=1.5cm,right=3cm,marginparwidth=1.75cm]{geometry}
\usepackage{siunitx}
\usepackage{amsmath}
\usepackage{tabto}
\usepackage{xcolor}
\usepackage{enumitem}
\usepackage[colorlinks=true, allcolors=blue]{hyperref}
\title{\raggedright Verifica di Matematica  2/1/2024}
\date{}
\begin{document}

\maketitle

\textbf{Esercizio 1 (Integrali Indefiniti)}:\\
\par $\int \tan x \,dx$ \\\\

\textit{[ Soluzione: $\int \tan x \,dx = -\ln |\cos x| + C$ ]}\\\\

\textbf{Esercizio 2 (Integrali Indefiniti)}:\\
\par $\int \sqrt{x} \,dx$ \\\\

\textit{[ Soluzione: $\int \sqrt{x} \,dx = \frac{2}{3}x^{3/2} + C$ ]}\\\\

\textbf{Esercizio 3 (Integrali Indefiniti)}:\\
\par $\int (2x + 3) \,dx$ \\\\

\textit{[ Soluzione: $\int (2x + 3) \,dx = x^2 + 3x + C$ ]}\\\\

\textbf{Esercizio 4 (Integrali Indefiniti)}:\\
\par $\int \sin^2 x \,dx$ \\\\

\textit{[ Soluzione: $\int \sin^2 x \,dx = \frac{1}{2}x - \frac{1}{4}\sin 2x + C$ ]}\\\\

\textbf{Esercizio 5 (Integrali Indefiniti)}:\\
\par $\int (x^2 - 2x) \,dx$ \\\\

\textit{[ Soluzione: $\int (x^2 - 2x) \,dx = \frac{1}{3}x^3 - x^2 + C$ ]}\\\\

\end{document}
