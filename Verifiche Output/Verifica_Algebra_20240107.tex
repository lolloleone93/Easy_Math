\documentclass{article}
\usepackage{amsmath}
\usepackage[italian]{babel}
\usepackage[letterpaper,top=1.5cm,bottom=2cm,left=1.5cm,right=3cm,marginparwidth=1.75cm]{geometry}
\usepackage{siunitx}
\usepackage{amsmath}
\usepackage{tabto}
\usepackage{xcolor}
\usepackage{enumitem}
\usepackage[colorlinks=true, allcolors=blue]{hyperref}
\title{\raggedright Verifica di Matematica  7/1/2024}
\date{}
\begin{document}

\maketitle

\textbf{Esercizio 1 (Integrali Indefiniti)}:\\
\par $\int (2x + 3)  dx$ \\\\

\textit{[ Soluzione: $\int (2x + 3)  dx = x^2 + 3x + C$ ]}\\\\

\textbf{Esercizio 2 (Integrali Indefiniti)}:\\
\par $\int (e^x - \sin x)  dx$ \\\\

\textit{[ Soluzione: $\int (e^x - \sin x)  dx = e^x + \cos x + C$ ]}\\\\

\textbf{Esercizio 3 (Integrali Indefiniti)}:\\
\par $\int \frac{1}{x}  dx$ \\\\

\textit{[ Soluzione: $\int \frac{1}{x}  dx = \ln |x| + C$ ]}\\\\

\textbf{Esercizio 4 (Integrali Indefiniti)}:\\
\par $\int \frac{1}{1 + x^2}  dx$ \\\\

\textit{[ Soluzione: $\int \frac{1}{1 + x^2}  dx = \arctan x + C$ ]}\\\\

\textbf{Esercizio 5 (Integrali Indefiniti)}:\\
\par $\int (x^2 - 2x)  dx$ \\\\

\textit{[ Soluzione: $\int (x^2 - 2x)  dx = \frac{1}{3}x^3 - x^2 + C$ ]}\\\\

\textbf{Esercizio 6 (Integrali Indefiniti)}:\\
\par $\int \sin^2 x  dx$ \\\\

\textit{[ Soluzione: $\int \sin^2 x  dx = \frac{1}{2}x - \frac{1}{4}\sin 2x + C$ ]}\\\\

\textbf{Esercizio 7 (Integrali Indefiniti)}:\\
\par $\int e^{2x}  dx$ \\\\

\textit{[ Soluzione: $\int e^{2x}  dx = \frac{1}{2}e^{2x} + C$ ]}\\\\

\textbf{Esercizio 8 (Integrali Indefiniti)}:\\
\par $\int \cos^2 (x)  dx$ \\\\

\textit{[ Soluzione: $\int \cos^2 (x)  dx = \frac{1}{2}x + \frac{1}{4}\sin 2x + C$ ]}\\\\

\textbf{Esercizio 9 (Integrali Indefiniti)}:\\
\par $\int \frac{e^x}{e^x + 1}  dx$ \\\\

\textit{[ Soluzione: $\int \frac{e^x}{e^x + 1}  dx = \ln |e^x + 1| + C$ ]}\\\\

\textbf{Esercizio 10 (Integrali Indefiniti)}:\\
\par $\int \frac{2}{x^2}  dx$ \\\\

\textit{[ Soluzione: $\int \frac{2}{x^2}  dx = -\frac{2}{x} + C$ ]}\\\\

\textbf{Esercizio 11 (Integrali Indefiniti)}:\\
\par $\int \sqrt{x}  dx$ \\\\

\textit{[ Soluzione: $\int \sqrt{x}  dx = \frac{2}{3}x^{3/2} + C$ ]}\\\\

\textbf{Esercizio 12 (Integrali Indefiniti)}:\\
\par $\int (\tan x)  dx$ \\\\

\textit{[ Soluzione: $\int (\tan x)  dx = -\ln |\cos x| + C$ ]}\\\\

\textbf{Esercizio 13 (Integrali Indefiniti)}:\\
\par $\int \frac{x^2}{1 + x^2}  dx$ \\\\

\textit{[ Soluzione: $\int \frac{x^2}{1 + x^2}  dx = \frac{1}{2}x - \frac{1}{2}\arctan x + C$ ]}\\\\

\textbf{Esercizio 14 (Integrali Indefiniti)}:\\
\par $\int \frac{1}{x^2 + 1}  dx$ \\\\

\textit{[ Soluzione: $\int \frac{1}{x^2 + 1}  dx = \arctan x + C$ ]}\\\\

\textbf{Esercizio 15 (Integrali Indefiniti)}:\\
\par $\int e^x \sin x  dx$ \\\\

\textit{[ Soluzione: $\int e^x \sin x  dx = \frac{1}{2}(e^x\sin x - e^x\cos x) + C$ ]}\\\\

\end{document}
