\documentclass{article}
\usepackage{amsmath}
\usepackage[italian]{babel}
\usepackage[letterpaper,top=1.5cm,bottom=2cm,left=1.5cm,right=3cm,marginparwidth=1.75cm]{geometry}
\usepackage{siunitx}
\usepackage{amsmath}
\usepackage{tabto}
\usepackage{xcolor}
\usepackage{enumitem}
\usepackage[colorlinks=true, allcolors=blue]{hyperref}
\title{\raggedright Verifica di Matematica  10/1/2024}
\date{}
\begin{document}

\maketitle

\textbf{Esercizio 1 (Integrali Casi Particolari)}:\\
\par $\int \sqrt{9-x^2}  dx$ \\\\

\textit{[ Soluzione: $\int \sqrt{9-x^2} dx = \int 3\sin^2(t) dt \text{ (con } x=3\sin(t)) =\frac{1}{2}(\sqrt{9-x^2}+9\arcsin {\frac{x}{3}}) + C$ ]}\\\\

\textbf{Esercizio 2 (Integrali Casi Particolari)}:\\
\par $\int \sqrt{1-x^2} dx$ \\\\

\textit{[ Soluzione: $\int \sqrt{1-x^2}  dx = \int \sin^2(t) dt \text{ (con } x=\sin(t))=\frac{1}{2}(\sqrt{1-x^2}+\arcsin x) + C$ ]}\\\\

\textbf{Esercizio 3 (Integrali Casi Particolari)}:\\
\par $\int \sqrt{x^2-1}  dx$ \\\\

\textit{[ Soluzione: $\int \sqrt{x^2-1}  dx = \int \sinh^2(t) dt \text{ (con } x=\cosh(t))=x\sqrt{x^2-1}-\int \frac{x^2}{\sqrt{x^2-1}} dx \text{(per parti)}=x\sqrt{x^2-1}-\int \frac{x^2-1}{\sqrt{x^2-1}} dx+\int \frac{1}{\sqrt{x^2-1}} dx=\frac{1}{2}(x \sqrt{x^2-1}-\log{\sqrt{x^2-1} + x}) + C$ ]}\\\\

\textbf{Esercizio 4 (Integrali Casi Particolari)}:\\
\par $\int \sqrt{x^2+1}  dx$ \\\\

\textit{[ Soluzione: $\int \sqrt{x^2+1}  dx = \int \sec^3(t) dt \text{ (con } x=\tan(t))=x\sqrt{x^2+1}-\int \frac{x^2}{\sqrt{x^2+1}} dx \text{(per parti)}=x\sqrt{x^2+1}-\int \frac{x^2+1}{\sqrt{x^2+1}} dx+\int \frac{1}{\sqrt{x^2+1}} dx=\frac{1}{2}(x\sqrt{x^2+1}+log{(x+\sqrt{x^2+1})}+ C$ ]}\\\\

\textbf{Esercizio 5 (Integrali Casi Particolari)}:\\
\par $\int \frac{1}{9x^2-6x+2}  dx$ \\\\

\textit{[ Soluzione: $\int \frac{1}{9x^2-6x+2}  dx=\int \frac{1}{(3x-1)^2+1} dx =\frac{1}{3}\arctan(3x-1)+ C$ ]}\\\\

\textbf{Esercizio 6 (Integrali Indefiniti)}:\\
\par $\int \frac{x^2}{1 + x^2}  dx$ \\\\

\textit{[ Soluzione: $\int \frac{x^2}{1 + x^2}  dx = \frac{1}{2}x - \frac{1}{2}\arctan x + C$ ]}\\\\

\textbf{Esercizio 7 (Integrali Indefiniti)}:\\
\par $\int \frac{1}{x^2 + 1}  dx$ \\\\

\textit{[ Soluzione: $\int \frac{1}{x^2 + 1}  dx = \arctan x + C$ ]}\\\\

\end{document}
