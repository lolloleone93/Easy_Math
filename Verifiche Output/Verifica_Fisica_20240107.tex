\documentclass{article}
\usepackage{amsmath}
\usepackage[italian]{babel}
\usepackage[letterpaper,top=1.5cm,bottom=2cm,left=1.5cm,right=3cm,marginparwidth=1.75cm]{geometry}
\usepackage{siunitx}
\usepackage{amsmath}
\usepackage{tabto}
\usepackage{xcolor}
\usepackage{enumitem}
\usepackage[colorlinks=true, allcolors=blue]{hyperref}
\title{\raggedright Verifica di Matematica  7/1/2024}
\date{}
\begin{document}

\maketitle

\textbf{Esercizio 1 (Moto Rettilineo Uniforme)}:\\
\par $\text{Un'auto si muove con velocit\`a costante di 20 m/s. Calcola la distanza percorsa in 3 minuti.}$ \\\\

\textit{[ Soluzione: $\text{Distanza} = \text{Velocit\`a} \cdot \text{Tempo} = 20 \, \text{m/s} \cdot (3 \times 60) \, \text{s}$ ]}\\\\

\end{document}
