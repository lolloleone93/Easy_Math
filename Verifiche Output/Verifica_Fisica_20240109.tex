\documentclass{article}
\usepackage{amsmath}
\usepackage[italian]{babel}
\usepackage[letterpaper,top=1.5cm,bottom=2cm,left=1.5cm,right=3cm,marginparwidth=1.75cm]{geometry}
\usepackage{siunitx}
\usepackage{amsmath}
\usepackage{tabto}
\usepackage{xcolor}
\usepackage{enumitem}
\usepackage[colorlinks=true, allcolors=blue]{hyperref}
\title{\raggedright Verifica di Matematica  9/1/2024}
\date{}
\begin{document}

\maketitle

\textbf{Esercizio 1 (Formule Inverse)}:\\
\par $\text{Trova la formula inversa per il calcolo del tempo (t) dalla formula della velocit\`a (v) nel moto uniformemente accelerato: } v = u + at$ \\\\

\textit{[ Soluzione: $\text{Formula Inversa: } t = \frac{v - u}{a}$ ]}\\\\

\textbf{Esercizio 2 (Formule Inverse)}:\\
\par $\text{Calcola il tempo (t) dalla formula della distanza (s) nel moto uniformemente accelerato: } s = ut + \frac{1}{2}at^2$ \\\\

\textit{[ Soluzione: $\text{Formula Inversa: } t = \sqrt{\frac{2s}{a}}$ ]}\\\\

\textbf{Esercizio 3 (Formule Inverse)}:\\
\par $\text{Trova la formula inversa per il calcolo della forza (F) dalla formula dell'accelerazione (a) nel secondo principio della dinamica: } F = ma$ \\\\

\textit{[ Soluzione: $\text{Formula Inversa: } a = \frac{F}{m}$ ]}\\\\

\textbf{Esercizio 4 (Formule Inverse)}:\\
\par $\text{Calcola la velocit\`a finale (v) dalla formula dell'energia cinetica (E\_k): } E\_k = \frac{1}{2}mv^2$ \\\\

\textit{[ Soluzione: $\text{Formula Inversa: } v = \sqrt{\frac{2E\_k}{m}}$ ]}\\\\

\textbf{Esercizio 5 (Formule Inverse)}:\\
\par $\text{Trova la formula inversa per il calcolo della massa (m) dalla formula dell'energia potenziale gravitazionale (U): } U = mgh$ \\\\

\textit{[ Soluzione: $\text{Formula Inversa: } m = \frac{U}{gh}$ ]}\\\\

\end{document}
